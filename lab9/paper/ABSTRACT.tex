%%%%%%%%%%%%%%%%%%%%%%%%%%%%%%%%%%%%%%%%%
%%            请在此填写摘要            %%
%% 请勿编译/排版此文件,请编译PAPER.tex!  %%
%%%%%%%%%%%%%%%%%%%%%%%%%%%%%%%%%%%%%%%%%
\begin{abstract}\small
	
	In the carbon cycle process, the decomposition of organic matter, especially the degradation of plant material and woody fibers involving fungi, is an essential link. We explore the relationship between fungal properties (growth rate and moisture resistance) and the decomposition rate of wood fibers to better understand the relationship and mechanism between the degradation of plant material and woody fibers and fungi. \par 
	Firstly, considering multiple factors, through data-based regression analysis, we initially establish a multiple linear regression model of the decomposition rate of wood fiber, and explain the rationality of the model from the perspective of biology and ecology. At the same time, from a mathematical point of view, we established a mathematical model containing an exponential relationship between fungal growth rate and fungal moisture resistance-the fungal growth rate model, which correlates these two biological characteristics of fungi. And the actual data verifies the rationality of this model. \par 
	Secondly, the decomposition of wood fibers by fungi in nature is often the result of the joint action of different populations and there are also interactions between fungal populations, Therefore, we use the growth rate and moisture resistance of fungi as the standard according to the K-Means clustering algorithm and the elbow rule to cluster fungi. In order to analyze the interactions within and between the fungal categories. Then, we optimized and improved the previously established fungal growth rate model based on the Monod equation in microbial dynamics and the theory of interaction between populations, and obtained a fungal growth rate model combined with interaction. We verify the rationality of the model by listing examples of interacting fungal populations under ideal conditions. The final result achieves the unity of theory and practice.\par 
	Thirdly, according to the optimized fungal growth model. We discuss and analyze the dynamics of the interactions in terms of long-term and short-term action time, revealing the principles of these two phenomena in nature. In addition, we analyzed the sensitivity of the model to rapidly changing natural conditions and tried to explain it from a biological perspective.\par 
	Then,in predicting the strengths and weaknesses of each fungal group and the likely combinations of fungi to survive, we considered the possible conditions in different climatic environment,including arid, semiarid, temperate, arboreal, and tropical rain forests.Combined with the analysis and experimental data of previous papers, we concluded that fungi with a high growth rate and moisture tolerance are more likely to survive when the environment is relatively stable.While when the weather is changeable,fungi with a large water niche width are more competent.We also further discussed the dependence between fungi and the environment and found out the possible seasonal changes of fungi in different areas.\par 
	Finally,our model successfully shows the relationship between fungal species diversity and decomposition rate. When the fungal species diversity is higher, the decomposition rate will also decrease. Because decomposition will emit carbon dioxide, it also indirectly implies the relationship between species diversity and carbon dioxide. It has positive significance for environmental issues related to greenhouse gas emission reduction, and is helpful to optimize the global carbon cycle and co-build a beautiful earth.\par 


    % 美赛论文中无需注明关键字。若您一定要使用,
    % 请将以下两行的注释号'%'去除,以使其生效;
    % 若您不使用,可直接将这段注释删除
    % \vspace{5pt}
    % \textbf{Keywords}: MATLAB, mathematics, LaTeX.

\end{abstract}




%%%%%%%%%%%%%%%%%%%%%%%%%%%%%%%%%%%%%%%%%%
% 如不理解以下部分中各命令的含义,请勿修改! %
%%%%%%%%%%%%%%%%%%%%%%%%%%%%%%%%%%%%%%%%%%

%---------以下生成sheet页----------
% 下面的语句可调整全文行距为标准值的0.6倍,请自行使用
% \renewcommand{\baselinestretch}{0.6}\normalsize
\maketitle  			% 生成sheet页
\thispagestyle{empty}   % 不要页眉页脚和页码
\setcounter{page}{-100} % 此命令仅是为了避免页码重复报错,不要在意

%---------以下生成目录----------
\newpage
\tableofcontents
\thispagestyle{empty}   % 不要页眉页脚和页码
\newpage

%---------以下生成正文----------
\setlength\parskip{0.8\baselineskip}  % 调整段间距
\setcounter{page}{1}    % 从正文开始计页码
\pagestyle{fancy}